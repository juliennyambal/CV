%% start of file 'template.tex'.
%% Copyright 2006-2013 Xavier Danaux (xdanaux@gmail.com).
%
% This work may be distributed and/or modified under the
% conditions of the LaTeX Project Public License version 1.3c,
% available at http://www.latex-project.org/lppl/.


\documentclass[letterpaper]{moderncv}        % possible options include font size ('10pt', '11pt' and '12pt'), paper size ('a4paper', 'letterpaper', 'a5paper', 'legalpaper', 'executivepaper' and 'landscape') and font family ('sans' and 'roman')
\usepackage{textcomp}
\usepackage{siunitx}
% moderncv themes
\moderncvstyle{classic}                             % style options are 'casual' (default), 'classic', 'oldstyle' and 'banking'
\moderncvcolor{blue}                               % color options 'blue' (default), 'orange', 'green', 'red', 'purple', 'grey' and 'black'
%\renewcommand{\familydefault}{\sfdefault}         % to set the default font; use '\sfdefault' for the default sans serif font, '\rmdefault' for the default roman one, or any tex font name
%\nopagenumbers{}                                  % uncomment to suppress automatic page numbering for CVs longer than one page

% character encoding
\usepackage[utf8]{inputenc}                       % if you are not using xelatex ou lualatex, replace by the encoding you are using
%\usepackage{CJKutf8}                              % if you need to use CJK to typeset your resume in Chinese, Japanese or Korean

\usepackage{natbib}
\usepackage{bibentry}
\nobibliography*
\bibliographystyle{apalike}

% adjust the page margins
\usepackage[scale=0.90]{geometry}
%\setlength{\hintscolumnwidth}{3cm}                % if you want to change the width of the column with the dates
%\setlength{\makecvtitlenamewidth}{10cm}           % for the 'classic' style, if you want to force the width allocated to your name and avoid line breaks. be careful though, the length is normally calculated to avoid any overlap with your personal info; use this at your own typographical risks...

\renewcommand{\footrulewidth}{0pt} % line at the bottom
% will have a thickness of 0 pts; ie, no line
\usepackage{fancyhdr}

\pagestyle{fancy}
\cfoot{Julien Nyambal}

    % Profile
\name{Julien Nyambal}{}
\address{301 Florida Park Complex\\ 78 Daniel Malan Avenue\\ Florida Park, Roodepoort, South Africa\\}
\phone[mobile]{00 27 83 398 9020}
\email{Julien.nyambal@gmail.com}
\extrainfo{%
	\href{https://github.com/juliennyambal}{GitHub} \\
	\href{https://www.linkedin.com/in/juliennyambal}{LinkedIn}\\
	\href{https://stackoverflow.com/users/2781787/julien-nyambal}{StackOverFlow}
}
\begin{document}
\makecvtitle

%\section{Personal Information}
%\cventry
%{DoB:}
%{13 February 1989}
%{}
%{}
%{}
%{}
%\cventry
%{Passport n$^\circ$}
%{0332231}
%{}
%{}
%{}
%{}
%\cventry
%{Nationality}
%{Cameroonian}
%{}
%{}
%{}
%{}
%
%\cventry
%{Gender}
%{Male}
%{}
%{}
%{}
%{}

\section{Education}
\cventry
{2019 - Present}
{PhD in Computer Science (Machine Learning). Supervisors: Dr. Richard Klein \& Prof. Turgay Celik}
{University of the Witwatersrand}
{}
{\textit{Johannesburg, South Africa}}
{}
\cventry
{2017 - 2018}
{Master of Science in Computer Science (Machine Learning - Computer Vision), Supervisor: Dr. Richard Klein}
{University of the Witwatersrand}
{}
{\textit{Johannesburg, South Africa}}
{}
\cventry
{2016}
{Bachelor of Science (Honours) in Computer Science}
{University of the Witwatersrand}
{Passed with Distinction}
{\textit{Johannesburg, South Africa}}
{}
\cventry
{2012 - 2014}
{Bachelor of Science in Computer Science \& Mathematics}
{University of Zululand}
{}
{\textit{Kwadlangezwa, South Africa}}
{}
\section{Professional Experience}

\cventry
{12/2019 -- Present}
{Software Engineer, Data Scientist}
{Entelect}
{Johannesburg, South Africa}
{}
{\begin{itemize}%
		\item \textbf{Standard Bank: Forex Customer Analysis}: Using time series to understand trends around the user bahaviour with regards to the Forex changes between the Rand (ZAR), the US Dollar (USD) and the Nigerian Naira (NGN)
		\item \textbf{Standard Bank: Internal Incident Reporting System RMP}: Adding the GraphDB feature to the system, and helping with creating a data flow between the data lake and the graphDB.
		\item \textbf{Standard bank: Internal Credit market Data}: My team and I wrote from ground up a system that pulls data from different forex sources to make it available to the bank for internal use. I mainly built the whole Python API base using Flask and Swagger. Thereafter build a docker replica image of the system. I wrote in Python different mathematical formulae use for some internal processes like the Dividend Yield. I also persist most of the transactions to both Mongo and SQL Server Management Studio (SSMS).
\end{itemize}}

\cventry
{02/2019 -- Present}
{Part Time Lecturer}
{University of the Witwatersrand}
{Johannesburg, South Africa}
{}
{\begin{itemize}%
		\item Introduction to Programming and Algorithms, COMS 1022A
		\item Introduction to Data Structures And Algorithms COMS 1017A
\end{itemize}}

\cventry
{04/2018 -- 12/2019}
{Software Developer, Data Scientist}
{Retro Rabbit}
{Johannesburg, South Africa}
{}
{\begin{itemize}%
		\item \textbf{\href{https://www.kalido.me/}{Kalido}}: My role was to writes python scripts to enhance the current machine learning model responsible for matching two requests that were english user sentences. The management of all the requests where handle using Spark mostly. I implemented the autoscaling infrastructure using AWS ECS and leaving the images in AWS ECR. That was to increase the faster ingestion of the request in production. 
		\item \textbf{Nedbank \href{https://www.nedbank.co.za/content/nedbank/desktop/gt/en/personal/nedbank-money/avo.html}{Avo by Nedbank}}: I was part of the data team and responsible for data analysis and data modelling. I dealt with millions of transactional data from the customers of Nedbank that helped to create user personas based on their spending habits using some machine learning algorithms like clustering. This allowed me to build a system to tailor the offers based on the users balance. I was mostly working in the back-end side of the app that includes the PowerBI dashboard for the app interactions, the different transactions between components using AWS Lambda, RDS and EC2.
\end{itemize}}

\cventry
{06/2017 -- 04/2018}
{Teaching Assistant}
{University of Witwatersrand}
{Johannesburg, South Africa}
{}
{\begin{itemize}%
	\item Marking assignments, tests, invigilations, tutoring, monitoring laboratory work
	\end{itemize}}
\cventry
{02/2013 -- 11/2014}
{Teaching Assistant}
{University of Zululand}
{Kwadlangezwa, South Africa}
{}
{\begin{itemize}%
	\item Marking assignments, tutoring, monitoring laboratory work
	\end{itemize}}
\section{Skills}
\cvitem{Languages}{Python(fluent), Golang (intermediate), Java, Scala(basic), C++(basic), \LaTeX, Shell/Bash (intermediate), MQL5 (Basic)}
\cvitem{Frameworks}{Keras ,Tensorflow, Git, PowerBI, Neo4J}
\cvitem{Databases}{Mongo, MySQL, SSMS, RDS, Redshift}
\cvitem{OS}{Ubuntu Linux(Prefered), Windows, MacOS(Basic)}
\cvitem{Others}{Docker, AWS(ECS, EC2), Kubernetes(basic), Hadoop(basic), Spark(Intermediate)}
			
\vspace{1mm}

\nobibliography{publication}

	
\section{Publications}
\cventry
{2017}
{\bibentry{8261114}}
{}
{}
{}
{}

\section{Training \& Certifications}

\cventry
{May 2020}
{AWS Certified Cloud Practitioner}
{\href{https://www.certmetrics.com/amazon/public/badge.aspx?i=9&t=c&d=2020-05-30&ci=AWS01387974}{Link to Certificate}}
{AWS}
{}
{}

\cventry
{May 2020}
{Introduction to Calculus}
{\href{https://www.coursera.org/account/accomplishments/verify/VNY949FASZA2}{Link to Certificate}}
{Coursera}
{}
{}

\cventry
{June 2017}
{High Performance Computing Winter School at Stellenbosch University: C \& Python programming for cluster computing.}
{Hosted by the CHPC}
{CSIR}
{}
{}

\section{Talks}
\cventry
{4 April 2018}
{Introduction to Machine Learning and Deep Learning: \href{http://www.deeplearningindaba.com/indabax-centrafrique.html}{Indaba X Cameroon 2018}}
{Deep learning IndabaX Centrafrique}
{Ecole Nationale Supérieure Polytechnique de Yaoundé - Cameroon}
{}
{}

\cventry
{7 September 2018}
{Computer vision in your daily life: \href{https://www.youtube.com/watch?v=0Z--tiJ3FyE}
	{Retro Rabbit Conference}}
{Retro Rabbit Conference}
{Retro Rabbit Pretoria - South Africa}
{}
{}

\cventry
{2 April 2019}
{Machine Learning in Production:    \href{https://indabaxcameroon.github.io}{Indaba X Cameroon 2019}}
{Deep learning IndabaX Cameroon}
{L'Institut Français du Cameroun - Cameroon}
{}
{}

\cventry
{3 May 2019}
{Deep Learning and its application: \href{https://sites.google.com/view/indabaxethiopia2019/speakers?authuser=0}{Indaba X Ethiopia 2019}}
{Deep learning IndabaX Ethiopia}
{Wolaita Sodo University - Ethiopia}
{}
{}

\cventry
{20 February 2020}
{Deep Learning and its application: \href{https://www.meetup.com/Johannesburg-Artificial-Intelligence-Meetup/events/268234198/}{Johannesburg // AI - Machine Learning - Chatbots}}
{Johannesburg // AI - Machine Learning - Chatbots}
{Johannesburg - South Africa}
{}
{}
	
\section{Extracurricular Activity}	
\cventry
{4 - 6 April 2018}
{Co-Organizer Deep learning IndabaX Centrafrique}
{Ecole Nationale Supérieure Polytechnique de Yaoundé - Cameroon}
{}
{}
{}

\cventry
{2 - 4 April 2019}
{Co-Organizer Deep learning IndabaX Cameroon}
{French Institute - Yaounde, Cameroon}
{}
{}
{}

\section{References}
\cventry
{1-}
{Dr. Richard Klein}
{}
{\textit{Lecturer}}
{}
{\textbf{Institution:} University of the Witwatersrand\\
	\textbf{Unit:} School of Computer Science and Applied Mathematics\\
	\textbf{E-mail:} Richard.Klein@wits.ac.za}
\vspace{1mm}

\cventry
{2-}
{Dr. Pravesh Ranchod}
{}
{\textit{Lecturer}}
{}
{\textbf{Institution:} University of the Witwatersrand\\
	\textbf{Unit:} School of Computer Science and Applied Mathematics\\
	\textbf{E-mail:} Pravesh.Ranchod@wits.ac.za}
\vspace{1mm}

\cventry
{3-}
{Prof. Turgay Celik}
{}
{\textit{Professor}}
{}
{\textbf{Institution:} University of the Witwatersrand\\
	\textbf{Unit:} School of Computer Science and Applied Mathematics\\
	\textbf{E-mail:} Turgay.Celik@wits.ac.za}
\vspace{1mm}

\cventry
{4-}
{Jade Abbott}
{}
{}
{}
{\textbf{Company:} Retro Rabbit\\
	\textbf{Role:} Machine Learning Engineer\\
	\textbf{E-mail:} Jabbott@retrorabbit.co.za}
\vspace{1mm}
\end{document}