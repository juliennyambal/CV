%% start of file 'template.tex'.
%% Copyright 2006-2013 Xavier Danaux (xdanaux@gmail.com).
%
% This work may be distributed and/or modified under the
% conditions of the LaTeX Project Public License version 1.3c,
% available at http://www.latex-project.org/lppl/.


\documentclass[letterpaper]{moderncv}        % possible options include font size ('10pt', '11pt' and '12pt'), paper size ('a4paper', 'letterpaper', 'a5paper', 'legalpaper', 'executivepaper' and 'landscape') and font family ('sans' and 'roman')
\usepackage{textcomp}
\usepackage{siunitx}
% moderncv themes
\moderncvstyle{classic}                             % style options are 'casual' (default), 'classic', 'oldstyle' and 'banking'
\moderncvcolor{blue}                               % color options 'blue' (default), 'orange', 'green', 'red', 'purple', 'grey' and 'black'
%\renewcommand{\familydefault}{\sfdefault}         % to set the default font; use '\sfdefault' for the default sans serif font, '\rmdefault' for the default roman one, or any tex font name
%\nopagenumbers{}                                  % uncomment to suppress automatic page numbering for CVs longer than one page

% character encoding
\usepackage[utf8]{inputenc}                       % if you are not using xelatex ou lualatex, replace by the encoding you are using
%\usepackage{CJKutf8}                              % if you need to use CJK to typeset your resume in Chinese, Japanese or Korean

\usepackage{natbib}
\usepackage{bibentry}
\nobibliography*
\bibliographystyle{apalike}

% adjust the page margins
\usepackage[scale=0.90]{geometry}
%\setlength{\hintscolumnwidth}{3cm}                % if you want to change the width of the column with the dates
%\setlength{\makecvtitlenamewidth}{10cm}           % for the 'classic' style, if you want to force the width allocated to your name and avoid line breaks. be careful though, the length is normally calculated to avoid any overlap with your personal info; use this at your own typographical risks...

\renewcommand{\footrulewidth}{0pt} % line at the bottom
% will have a thickness of 0 pts; ie, no line
\usepackage{fancyhdr}
\pagestyle{fancy}
\cfoot{Julien Nyambal}

    % Profile
\name{Julien Nyambal}{}
\address{301 Florida Park Complex\\ 78 Daniel Malan Avenue\\ Florida Park, Roodepoort, South Africa\\}
\phone[mobile]{00 27 83 398 9020}
\email{Julien.nyambal@gmail.com}
\extrainfo{%
	\href{https://github.com/juliennyambal}{GitHub} \\
	\href{https://www.linkedin.com/in/juliennyambal}{LinkedIn}\\
	\href{https://stackoverflow.com/users/2781787/julien-nyambal}{StackOverFlow}
}
    \begin{document}
\makecvtitle

%\section{Personal Information}
%\cventry
%{DoB:}
%{13 February 1989}
%{}
%{}
%{}
%{}
%\cventry
%{Passport n$^\circ$}
%{0332231}
%{}
%{}
%{}
%{}
%\cventry
%{Nationality}
%{Cameroonian}
%{}
%{}
%{}
%{}
%
%\cventry
%{Gender}
%{Male}
%{}
%{}
%{}
%{}

\section{Education}
\cventry
{2019 - Present}
{PhD in Computer Science (Machine Learning), Supervisor: Dr. Richard Klein}
{University of the Witwatersrand}
{}
{\textit{Johannesburg, South Africa}}
{}
\cventry
{2017 - 2018}
{Master of Science in Computer Science (Machine Learning - Computer Vision), Supervisor: Dr. Richard Klein}
{University of the Witwatersrand}
{}
{\textit{Johannesburg, South Africa}}
{}
\cventry
{2016}
{Bachelor of Science (Honours) in Computer Science}
{University of the Witwatersrand}
{Passed with Distinction}
{\textit{Johannesburg, South Africa}}
{}
\cventry
{2012 - 2014}
{Bachelor of Science in Computer Science \& Mathematics}
{University of Zululand}
{}
{\textit{Kwadlangezwa, South Africa}}
{}
\section{Professional Experience}

\cventry
{04/2018 -- Present}
{Software Developer, Data Scientist}
{Retro Rabbit}
{Johannesburg, South Africa}
{}
{\begin{itemize}%
		\item \textbf{Automatic Parking Space Detection}: Improving the Honours project for automatic detection and comparison between Support Vector Machines (SVM) and Convolutional Neural Networks (CNN). We noticed that joining CNN as feature extractor and the SVM as classifier (linear or RBF) was slightly better than the CNN.
		\item \textbf{\href{https://www.kalido.me/}{Kalido}}: My role was to writes python scripts to enhance the current machine learning model responsible for matching two requests that where english user sentences. The management of all the requests where handle using Spark mostly. I implemeted the autoscaling infrastructure using ECS and leaving the images in ECR. That was to increase the faster ingestion of the request in production. 
		\item \textbf{\href{http://www.marketedge.nedbank.co.za/}{Market Edge 3.0}}: I am currently in the data team and responsible for data analysis and data modeling. I deal with millions of transactional data that will help me to create user personas based on their spending habit using some machine learning algorithms like clustering.
\end{itemize}}

\cventry
{02/2019 -- Present}
{Part Time Lecturer}
{University of the Witwatersrand}
{Johannesburg, South Africa}
{}
{\begin{itemize}%
		\item Introduction to Programming and Algorithms
		\item Introduction to Data Structures And Algorithms 
	\end{itemize}}

\cventry
{06/2017 -- 04/2018}
{Teaching Assistant}
{University of Witwatersrand}
{Johannesburg, South Africa}
{}
{\begin{itemize}%
	\item Marking assignments, tests, invigilations, tutoring, monitoring laboratory work
	\end{itemize}}
\cventry
{02/2013 -- 11/2014}
{Teaching Assistant}
{University of Zululand}
{Kwadlangezwa, South Africa}
{}
{\begin{itemize}%
	\item Marking assignments, tutoring, monitoring laboratory work
	\end{itemize}}
\section{Skills}
\cvitem{Languages}{Python(fluent), SQL, Golang (intermediate), Java, Scala(basic), C++(basic), \LaTeX, Shell/Bash (intermediate)}
\cvitem{Frameworks}{Keras ,Tensorflow, Git}
\cvitem{OS}{Ubuntu Linux(Prefered), Windows (intermediate), MacOS(basic)}
\cvitem{Others}{Docker, AWS(ECS, EC2), Kubernetes(basic), Hadoop(basic), Spark(Intermediate)}
			
\vspace{1mm}

\nobibliography{publication}

	
\section{Publications}
\cventry
{2017}
{\bibentry{8261114}}
{}
{}
{}
{}

\section{Training}
\cventry
{2017}
{High Performance Computing Winter School at Stellenbosch University: C \& Python programming for cluster computing.}
{Hosted by the CHPC}
{CSIR}
{}
{}

\section{Talks}
\cventry
{4 April 2018}
{Introduction to Machine Learning and Deep Learning}
{Deep learning IndabaX Centrafrique}
{Ecole Nationale Supérieure Polytechnique de Yaoundé - Cameroon}
{}
{}

\cventry
{7 September 2018}
{Computer vision in your daily life}
{Retro Rabbit Conference}
{Retro Rabbit Pretoria - South Africa}
{}
{}

\cventry
{2 April 2019}
{Machine Learning in Production}
{Deep learning IndabaX Cameroon}
{L'Institut Français du Cameroun - Cameroon}
{}
{}

\cventry
{3 May 2019}
{Deep Learning and its application}
{Deep learning IndabaX Ethiopia}
{Wolaita Sodo University - Ethiopia}
{}
{}
	
\section{Extracurricular Activity}	
\cventry
{4 - 6 April 2018}
{Co-Organizer Deep learning IndabaX Centrafrique}
{Ecole Nationale Supérieure Polytechnique de Yaoundé - Cameroon}
{}
{}
{}

\cventry
{2 - 4 April 2019}
{Co-Organizer Deep learning IndabaX Cameroon}
{French Institute - Yaounde, Cameroon}
{}
{}
{}

\section{References}
\cventry
{1-}
{Dr. Richard Klein}
{}
{\textit{Lecturer}}
{}
{\textbf{Institution:} University of the Witwatersrand\\
	\textbf{Unit:} School of Computer Science and Applied Mathematics\\
	\textbf{E-mail:} Richard.Klein@wits.ac.za}
\vspace{1mm}

\cventry
{2-}
{Dr. Pravesh Ranchod}
{}
{\textit{Lecturer}}
{}
{\textbf{Institution:} University of the Witwatersrand\\
	\textbf{Unit:} School of Computer Science and Applied Mathematics\\
	\textbf{E-mail:} Pravesh.Ranchod@wits.ac.za}
\vspace{1mm}

\cventry
{3-}
{Prof. Turgay Celik}
{}
{\textit{Professor}}
{}
{\textbf{Institution:} University of the Witwatersrand\\
	\textbf{Unit:} School of Computer Science and Applied Mathematics\\
	\textbf{E-mail:} Turgay.Celik@wits.ac.za}
\vspace{1mm}
\end{document}